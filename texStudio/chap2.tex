\chapter{Математичний апарат для вирішення задачі оптимізації}

Оптимізація ресурсів у Kubernetes спирається на математичні дисципліни для моделювання, аналізу та пошуку рішень.

\section{Огляд ключових математичних дисциплін та їх застосування в контексті Kubernetes}

\subsection{Теорія оптимізації (лінійне та цілочисельне програмування)}
Лінійне (ЛП) та цілочисельне (ЦЛП) програмування формалізують задачі розподілу ресурсів, наприклад, вибір вузлів для мінімізації витрат за обмежень на ресурси. Цільова функція може виглядати так:
$$ \text{Minimize } Z = \sum_{i=1}^{N} c_i x_i $$
де $c_i$ – вартість i-го типу вузла, а $x_i$ – кількість вузлів i-го типу. Обмеження можуть включати задоволення сумарних потреб у ресурсах, наприклад, для ЦП:
$$ \sum_{i=1}^{N} R_{cpu,i} x_i \geq D_{cpu} $$
де $R_{cpu,i}$ – кількість ЦП, що надається вузлом i-го типу, а $D_{cpu}$ – загальна потреба в ЦП.
Опукла оптимізація використовується для складніших розподілів, де моделі можуть враховувати матрицю складу ресурсів $K_{ri}$ (кількість ресурсу r, що надається інстансом типу i) та вектор попиту $d_r$ для кожного ресурсу r:
$$ \sum_{i=1}^{N} K_{ri} x_i \geq d_r \quad \forall r $$
Ці методи лежать в основі right-sizing та bin packing.

\subsection{Статистичний аналіз та теорія ймовірностей}
Статистика та теорія ймовірностей необхідні для аналізу даних використання ресурсів та характеризування навантажень. Розрахунок середніх значень, стандартних відхилень, процентилів ($P_{95}$) використовується для right-sizing. Формула для середнього:
$$ \mu = \frac{1}{n} \sum_{i=1}^{n} x_i $$
А для дисперсії (квадрата стандартного відхилення):
$$ \sigma^2 = \frac{1}{n} \sum_{i=1}^{n} (x_i - \mu)^2 $$
Аналіз часових рядів (ARIMA, Prophet) – для прогнозування навантаження та проактивного масштабування. Наприклад, модель Prophet може бути виражена як:
$$ y(t) = g(t) + s(t) + h(t) + \epsilon_t $$
де $g(t)$ – тренд, $s(t)$ – сезонність, $h(t)$ – ефекти свят, а $\epsilon_t$ – похибка.

\subsection{Теорія графів та дискретна математика}
Теорія графів моделює задачі розміщення подів, наприклад, podAntiAffinity як розфарбовування графа. Задачі оптимізації потоків (min-cost max-flow) застосовуються у просунутих планувальниках (Firmament) для оптимального розміщення. Ці методи зазвичай описуються алгоритмічно, а не простими загальними формулами, придатними для короткого викладу.

\subsection{Чисельні методи та математичне моделювання}
Математичне моделювання створює абстрактні представлення системи Kubernetes для аналізу "що, якщо". Чисельні методи розв'язують складні моделі. Додавання буфера (напр., 15\%) є евристикою, що може бути обґрунтована складнішими моделями. Наприклад, розрахунок потреби з буфером:
$$ \text{Required}_{\text{buffer}} = \text{Required}_{\text{base}} \times (1 + \text{Percent buffer}) $$