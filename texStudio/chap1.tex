\chapter{Актуальність задачі оптимізації ресурсів у Kubernetes кластерах}

\section{Техніко-економічні аспекти проблеми}

Актуальність оптимізації ресурсів у Kubernetes зумовлена економічною ефективністю (значні витрати на хмарні ресурси, марнотратство до 47\% бюджетів через неефективність), продуктивністю та надійністю додатків (недостатнє виділення ресурсів призводить до збоїв), масштабованістю системи (ефективне використання ресурсів дозволяє точніше масштабувати), екологічним аспектом (менше ресурсів – менший вуглецевий слід) та складністю сучасних додатків (ШІ/МН вимагають витонченого управління ресурсами). Оптимізація є комплексною проблемою на перетині технологій, економіки та бізнес-цілей.

\section{Огляд існуючих підходів та інструментів}

Існують вбудовані механізми Kubernetes, інструменти моніторингу/візуалізації та сторонні/власні рішення.

\subsection{Вбудовані механізми Kubernetes: Основні функції}
Kubernetes надає:
\begin{itemize}
    \item \textbf{Запити (Requests) та Ліміти (Limits):} Гарантована та максимальна кількість ресурсів для контейнера.
    \item \textbf{Класи Якості Обслуговування (QoS Classes):} Guaranteed, Burstable, BestEffort – визначають пріоритет поду.
    \item \textbf{Автоскейлери (HPA, VPA, CA):} Горизонтальний (HPA) масштабує кількість реплік, Вертикальний (VPA) оптимізує запити/ліміти подів, Автоскейлер Кластера (CA) регулює кількість вузлів. Їх узгоджене налаштування є складним завданням.
    \item \textbf{Стратегії Розміщення Подів:} nodeSelector, Affinity/AntiAffinity, Taints and Tolerations для контролю розміщення.
\end{itemize}

\subsection{Інструменти моніторингу та візуалізації: Головне призначення}
Для аналізу даних використовуються:
\begin{itemize}
    \item \textbf{Metrics Server:} Базовий агрегатор метрик для kubectl top та HPA.
    \item \textbf{Prometheus:} Стандарт для збору метрик, надає мову запитів PromQL.
    \item \textbf{Grafana:} Платформа для візуалізації даних, часто з Prometheus.
\end{itemize}
Налаштування Prometheus/Grafana вимагає експертизи, що створює нішу для простіших інструментів аналізу на основі kubectl.

\subsection{Сторонні рішення та розробка власних інструментів: Сутність підходів}
\begin{itemize}
    \item \textbf{Сторонні рішення:} Kubecost, Densify, CAST AI, Goldilocks, StormForge пропонують розширений аналіз, right-sizing, часто з використанням МН.
    \item \textbf{Розробка власних комплексів:} Дозволяє реалізувати специфічну логіку, як у даній роботі, фокусуючись на аналізі статичних даних kubectl для попереднього аудиту.
\end{itemize}