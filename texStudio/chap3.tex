\chapter{Результати: Аналіз та оптимізація за допомогою розробленого програмного комплексу}

Розроблений програмний комплекс аналізує конфігурації Kubernetes та надає рекомендації, спираючись на логіку, що концептуально відповідає математичним підходам.

\section{Опис розробленого програмного комплексу}
Програмний комплекс (скрипт Python) аналізує статичні файли з виводом команд kubectl для виявлення проблем конфігурації та надання рекомендацій, зокрема щодо вибору типів вузлів.
Ключові функції:
\begin{itemize}
	\item \textbf{Збір даних:} Парсинг виводів `kubectl describe nodes/pods`, `kubectl get pods -o wide`.
	\item \textbf{Агрегація та аналіз:} Визначення доступних ресурсів, запитів/лімітів, сумарних показників на вузлах.
	\item \textbf{Виявлення проблем:} Ідентифікація CPU/Memory Limits Overcommit, подів з високими запитами, подів без лімітів/запитів за допомогою евристик.
	\item \textbf{Генерація рекомендацій:} Попередження, розрахунок сумарних потреб (з буфером, напр., 15%), рекомендація одиничного оптимального типу інстансу (напр., AWS EC2) на основі CSV-файлу з характеристиками та вартістю.
\end{itemize}

\begin{table}[h!]
	\centering
	\caption{Порівняння розробленого програмного комплексу з іншими підходами}
	\label{tab:comparison}
	\footnotesize
	\begin{tabular}{@{}l >{\raggedright\arraybackslash}p{0.18\linewidth} >{\raggedright\arraybackslash}p{0.18\linewidth} >{\raggedright\arraybackslash}p{0.18\linewidth} >{\raggedright\arraybackslash}p{0.22\linewidth}@{}}
		\toprule
		\textbf{Характеристика} & \makecell[l]{\textbf{РПК}} & \makecell[l]{\textbf{Prometheus +}\\\textbf{Grafana}} & \makecell[l]{\textbf{Kubernetes}\\\textbf{VPA}} & \makecell[l]{\textbf{Комерційні}\\\textbf{рішення}} \\ 
		\midrule
		Джерело даних & Статичні файли (kubectl) & Метрики реального часу & Метрики реального часу, історичні дані & Метрики реального часу, дані про витрати, історичні дані \\
		\addlinespace
		Основний фокус & Аудит конфігурацій, базові помилки, початкові рекомендації & Моніторинг, візуалізація, алертинг & Автоматичне right-sizing подів & Комплексна оптимізація витрат, right-sizing, прогнозування (МН) \\
		\addlinespace
		Аналіз споживання & Відсутній (лише аналіз запитів/лімітів) & Детальний аналіз реального споживання & Аналіз реального споживання для подів & Глибокий аналіз реального споживання, часто з МН \\
		\addlinespace
		Автоматизація дій & Лише рекомендації & Немає (потребує дій або інтеграції) & Можливе автоматичне застосування змін & Можлива автоматизація, інтеграція з CI/CD \\
		\addlinespace
		Простота використання & Відносно простий (скрипт) & Потребує налаштування та експертизи & Інтегрується в Kubernetes & Залежить від рішення, надають UI та підтримку \\
		\bottomrule
	\end{tabular}
\end{table}


\section{Аналіз даних та ідентифікація проблем (на прикладі Log2\_1.txt)}
Аналіз Log2\_1.txt (Додаток А) показує:
\begin{itemize}
	\item \textbf{Аналіз по вузлах:}
	\begin{itemize}
		\item Вузол `ip-100-100-31-113...`: CPU Limits Overcommit 153.1% (ризик CPU throttling), под `xenia-stg/xenia-...-6hdql` з високими запитами CPU (51.5% ресурсів вузла).
		\item Вузол `ip-100-100-41-106...`: Под `xenia-stg/xenia-stg-mongodb-0` без лімітів CPU/пам'яті та з високими запитами CPU (52.3% ресурсів вузла).
	\end{itemize}
	\item \textbf{Загальний аналіз та рекомендація інстансу}:
	\begin{itemize}
		\item Сумарні запити кластера (CPU=3.03 cores, Пам'ять=2.25 GiB) + 15% буфер (потреби: CPU ≈ 3.48 cores, Пам'ять ≈ 2.59 GiB).
		\item Рекомендовано інстанс `a1.xlarge` (4 vCPU, 8 GiB RAM, \$0.1020/год) як найекономічніший одиничний варіант.
	\end{itemize}
\end{itemize}

\begin{table}[h!]
	\centering
	\caption{Ключові проблеми, виявлені програмним комплексом, та їх математична інтерпретація}
	\label{tab:problems_interpretation}
	\begin{tabular}{@{}p{0.45\linewidth} p{0.45\linewidth}@{}}
		\toprule
		\textbf{Проблема, виявлена РПК} & \textbf{Математична/концептуальна інтерпретація} \\ 
		\midrule
		CPU Limits Overcommit (153.1\%) & Порушення обмеження ресурсів вузла. Аналогічно до перевірки обмежень в задачах лінійного програмування. \\ 
		\addlinespace
		Под з високими запитами CPU (>50\% вузла) & Статистична аномалія (викид), що може вказувати на неефективне використання ресурсів або проблеми з "bin packing". \\ 
		\addlinespace
		Под без встановлених лімітів CPU/пам'яті & Відсутність верхніх меж для змінних споживання ресурсів в оптимізаційній моделі. \\ 
		\addlinespace
		Рекомендація типу інстансу (a1.xlarge) & Розв'язання спрощеної задачі оптимізації: min(вартість) за умов CPU\_інстансу $\geq$ потреба\_CPU та Пам’ять\_інстансу $\geq$ потреба\_Пам’яті. \\ 
		\bottomrule
	\end{tabular}
\end{table}


\section{Оцінка впливу запропонованих оптимізацій}
Виправлення виявлених проблем покращує стабільність, продуктивність та економічність.
\begin{itemize}
	\item \textbf{Підвищення стабільності/продуктивності:} Встановлення лімітів (напр., для `xenia-stg-mongodb-0`) запобігає OOMKilled. Коригування CPU Limits Overcommit знижує ризик CPU throttling. Оптимізація подів з "високими запитами" покращує "упаковку" подів (bin packing).
	\item \textbf{Оптимізація витрат:} Рекомендація інстансу `a1.xlarge` ($0.1020/год) є спрощеною, але спонукає до аналізу. Поточні вузли коштують більше ($0.2256/год). "Right-sizing" завищених запитів може дозволити Cluster Autoscaler зменшити кількість вузлів.
\end{itemize}

\section{Приклад оптимізації YAML-конфігурації (Додаток Б)}
Додаток Б показує оптимізований YAML для StatefulSet `xenia-stg-mongodb`. Початковий стан: запит CPU 1010m, пам'яті 0; ліміти 0.
Оптимізований YAML включає:
\begin{itemize}
	\item Встановлення лімітів: `limits: cpu: "1", memory: "2Gi"` для стабільності.
	\item Коригування запитів CPU: `requests: cpu: "500m"`.
	\item Встановлення запитів/лімітів Пам'яті: `requests: memory: "1Gi", limits: memory: "2Gi"`.
\end{itemize}
Для баз даних часто рекомендується `requests = limits` для QoS Guaranteed.

\section{Висновки щодо програмного комплексу}
Програмний комплекс – це інструмент для первинного аудиту статичних конфігурацій.
\begin{itemize}
	\item \textbf{Сильні сторони:} Автоматизація обробки даних `kubectl`, ідентифікація поширених помилок, базові рекомендації.
	\item \textbf{Обмеження:} Статичний аналіз (без реального споживання), спрощені рекомендації, відсутність МН.
	\item \textbf{Напрямки розвитку:} Інтеграція з Prometheus, розширені статистичні методи, МН, формальні оптимізаційні моделі.
\end{itemize}