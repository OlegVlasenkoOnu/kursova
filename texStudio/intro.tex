\chapter*{Вступ}
\addcontentsline{toc}{chapter}{Вступ}

\section*{Роль Kubernetes \\та проблема оптимізації ресурсів}

Kubernetes (K8s) став стандартом оркестрації контейнерів, автоматизуючи розгортання та управління додатками в мікросервісних архітектурах. Зі зростанням складності систем, особливо з ресурсоємними завданнями ШІ/МН, ефективне управління ресурсами (ЦП, ОЗП) стає критичним. Неоптимальне використання призводить до фінансових втрат через надлишкове виділення (overprovisioning), зниження продуктивності через недостатнє виділення (underprovisioning) (CPU throttling, OOMKilled), та проблем масштабованості. Конфігураційна оптимізація, зокрема правильне налаштування запитів (requests) та лімітів (limits), є ключовою, але часто недооцінюється, особливо враховуючи, що Kubernetes за замовчуванням не обмежує ресурси подів без явних лімітів.

\section*{Постановка задачі дослідження}

Проблема дослідження – виявлення неефективного використання ресурсів у Kubernetes через неоптимальні конфігурації та розробка підходів до їх оптимізації. Це включає ідентифікацію проблемних подів/вузлів, аналіз впливу на вартість та формування рекомендацій.

Мета – дослідити методи оптимізації та продемонструвати можливості розробленого програмного комплексу для аналізу конфігурацій на основі даних kubectl.

Завдання охоплюють аналіз актуальності, огляд інструментів, визначення математичного апарату, опис комплексу, аналіз його результатів, оцінку рішень та висновки.

Об'єкт – процеси управління ресурсами, предмет – методи аналізу конфігурацій за допомогою розробленого комплексу. Складність Kubernetes може ускладнити розуміння того, де саме можливе скорочення витрат.